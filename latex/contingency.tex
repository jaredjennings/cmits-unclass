% CMITS - Configuration Management for Information Technology Systems
% Based on <https://github.com/afseo/cmits>.
% Copyright 2015 Jared Jennings <mailto:jjennings@fastmail.fm>.
%
% Licensed under the Apache License, Version 2.0 (the "License");
% you may not use this file except in compliance with the License.
% You may obtain a copy of the License at
%
%    http://www.apache.org/licenses/LICENSE-2.0
%
% Unless required by applicable law or agreed to in writing, software
% distributed under the License is distributed on an "AS IS" BASIS,
% WITHOUT WARRANTIES OR CONDITIONS OF ANY KIND, either express or implied.
% See the License for the specific language governing permissions and
% limitations under the License.
\chapter{Contingency}
\label{Contingency}

\section{Contingency procedures}

A contingency has happened; one or more workstations or servers
must be reconstituted. You have these options:

\begin{itemize}

\item If you're building up one host in a temporary situation, it may be
simplest to go through this policy, manually implementing its requirements
on the machine in question. If you're not in the usual production setting
(e.g., filers are missing, networking to another building is out), you may
not want to follow the policy exactly, and when manually rebuilding, you
don't have to.

\item If you're setting up more than one host, or running for a while,
it's probably easier to set up a \emph{puppetmaster} and maybe a kickstart
server; this way, the hosts will implement the policy themselves, which is
faster and less error-prone.

\end{itemize}

We'll discuss the first alternative here; the second is the same as normal
production usage, which is detailed in~\S\ref{Production}
and~\S\ref{Procedures}.

If you'll be reading through this policy and manually applying it to a
machine, you'll need to know the syntax and semantics of the policy. In
general, refer to \cite{pro-puppet-book} and
\cite{puppet-generated-references}. A few salient specifics follow.

Start with {\tt nodes.pp} (\S\ref{manifests/nodes.pp}). Find the node
declaration for the host you are concerned with. Follow references from
there to high-level classes in {\tt templates.pp}
(\S\ref{manifests/templates.pp}), thence to the modules, where you will
find the details of how the host is configured. Some pieces of the policy
act based on \emph{facts} about the host, like \verb!$::hostname! or
\verb!$::kernelrelease!. You'll have to deduce the values of these facts
yourself and act accordingly.

Whichever way you choose, you must still personally follow the procedures
in \S\ref{Procedures}.

\section{ Contingency preparedness }

Some parts of this policy detail the ways that hosts under this
\CMITSPolicy\ should prepare for contingency situations:
\S\ref{module_contingency_backup}, \S\ref{module_log}.
