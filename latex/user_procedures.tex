% CMITS - Configuration Management for Information Technology Systems
% Based on <https://github.com/afseo/cmits>.
% Copyright 2015 Jared Jennings <mailto:jjennings@fastmail.fm>.
%
% Licensed under the Apache License, Version 2.0 (the "License");
% you may not use this file except in compliance with the License.
% You may obtain a copy of the License at
%
%    http://www.apache.org/licenses/LICENSE-2.0
%
% Unless required by applicable law or agreed to in writing, software
% distributed under the License is distributed on an "AS IS" BASIS,
% WITHOUT WARRANTIES OR CONDITIONS OF ANY KIND, either express or implied.
% See the License for the specific language governing permissions and
% limitations under the License.
\chapter{Procedures for users}
\label{ProceduresForUsers}

This chapter contains directions for users of hosts covered by this
\CMITSPolicy .



\section{Security Features User's Guide}
\label{SFUG}

This section contains guidance on the security features of information
systems under this \CMITSPolicy\ as required by the SPAN STIG.

\subsection{Single-user KVM switches}

Single-user keyboard-video-mouse (KVM) switches are used on unclassified
systems to reduce clutter due to too many keyboards, mice and monitors on
a desk. \documents{spanstig}{KVM01.002.00} Here's what you need to know
about how to operate these KVM switches securely:

\begin{enumerate}
\item Before interacting with a system connected to a KVM switch, make
    sure it's the system you think it is, and verify its classification.
    It should have a banner that lets you know this information.
\item Before switching to another system, lock your screen; then verify
    the identity and classification of the system you've switched to
    before interacting with it.
\end{enumerate}

\doneby{users}{spanstig}{KVM01.004.00} Do not connect a keyboard with a
smartcard reader to a KVM switch.

\doneby{users}{spanstig}{KVM01.005.00} Do not connect a wireless keyboard
or mouse to a KVM switch. \spanstig{KVM01.005.00} says that such devices
must comply with the current Wireless STIG, and the current Wireless STIG
says there are presently no compliant devices. (In order for them to be
compliant, they would have to use FIPS 140-2 compliant encryption.)


\subsection{Removable devices: prohibitions and requirements}

Here are some DoD-level requirements that you, the user, should know
about.

\doneby{users}{spanstig}{USB00.001.00} When removing a hot-swappable
device such as a USB device from one computer and connecting it to
another, you must wait at least 60 seconds in between.

\doneby{users}{spanstig}{USB01.001.00} MP3 players, camcorders and digital
cameras must not be attached to information systems (ISes) without prior
DAA approval.

\doneby{users}{spanstig}{USB01.002.00} No USB device may be connected to a
DoD IS unless approved by the Information Assurance Officer (IAO).

\documents{spanstig}{USB01.003.00} Thumb drives that look like anything
else besides a thumb drive (e.g., a watch, a pen, a piece of sushi, a
little teddy bear...) are not permitted and will be confiscated.

\doneby{users}{spanstig}{USB01.008.00} Any USB device with persistent
memory (e.g., USB hard drives) must be formatted with a modern filesystem
(e.g., NTFS, ext3, HFS; not FAT).


\subsection{USB usage and handling}

\documents{spanstig}{USB01.009.00} Existence of this section is required
by \spanstig{USB01.009.00}. \documents{spanstig}{USB01.010.00} Discussion
within this section of USB devices with persistent nonremovable memory is
required by \spanstig{USB01.010.00}.

Under current directives, you should not plug any USB storage device into
any host without authorization from the Information Assurance Manager
(IAM), authorization that is specific to you, the computer in question,
and the storage device in question.


\section{Miscellaneous prohibitions}

\documents{iacontrol}{IAIA-1,IAIA-2}\documents{databasestig}{DG0068}%
When using the \verb!psql! client to connect to the PostgreSQL database,
do not supply on the command line a conninfo string containing a password.
(Conninfo strings are described in the libpq documentation; try
this URL:
\url{http://www.postgresql.org/docs/8.4/static/libpq-connect.html}.) This
requirement flows from the more general requirement that database
passwords must not be stored in clear text.
